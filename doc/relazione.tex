\documentclass[12pt,a4paper]{report}

\usepackage[T1]{fontenc}   
\usepackage[utf8]{inputenc}
\usepackage[italian]{babel}
\usepackage{times} % Selected font
\usepackage{xcolor} % Colors
\usepackage[colorlinks=true, linkcolor=blue, citecolor=blue, urlcolor=blue, pdfborder={0 0 0}]{hyperref}
\usepackage{graphicx} % Images
\graphicspath{{img/}}
\usepackage{setspace} % For flexible line spacing management
\usepackage[margin=1in]{geometry} % Page geom
\usepackage{lastpage} % To track the total number of pages
\usepackage{fancyhdr} % To define a custom page numbering style
\usepackage[backend=biber, style=alphabetic, sorting=ydnt]{biblatex} % To include bibliography
\addbibresource{riferimenti.bib}

\setlength{\parskip}{1em}
\setlength{\parindent}{0pt}
\usepackage{enumitem} % To avoid interline between bullets and text before them
\setlist[itemize]{topsep=0pt, partopsep=0pt, parsep=0pt}

\begin{document}

\begin{spacing}{1.5}
\begin{titlepage}

\begin{center}

% marchio di ateneo
\includegraphics[width=6.5cm,height=4.7cm]{marchio-di-ateneo}

\vspace{4mm}

{Dipartimento di Informatica - Scienza e Ingegneria - DISI} 

\vspace{2mm}

{\large{Corso di Laurea in Ingegneria e Scienze Informatiche}}

\vspace{10mm}

{\huge{\bf{IMPLEMENTAZIONE CUDA DI}}}\\
\vspace{3mm}
{\huge{\bf{UN ALGORITMO DI PROIEZIONE}}}\\
\vspace{3mm}
{\huge{\bf{TOMOGRAFICA}}}\\
\vspace{3mm}

\vspace{5mm}
{Tesi di Laurea in Ingegneria e Scienze Informatiche}

\end{center}

\vspace{10mm}

\noindent\begin{minipage}[t]{0.40\textwidth}
{\large{Relatore: \\ Prof. Moreno Marzolla}}

\vspace{3mm}

{\large{Correlatore: \\ Prof. Elena Loli Piccolomini}}
\end{minipage}
\hfill
\begin{minipage}[t]{0.40\textwidth}\raggedleft
{\large{Presentata da: \\ Enrico Marchionni}}
\end{minipage}

\vfill

\hrule height 0.6mm

\begin{center}
{Sessione marzo 2025\\}
{Anno Accademico 2024/2025\\}
\end{center}

\end{titlepage}
\end{spacing}

\begin{abstract}
La Tomografia Computerizzata (TC) è una tecnica diagnostica che sfrutta le radiazioni ionizzanti (o raggi X) per ottenere
un'immagine tridimensionale di un'entità analizzata.
Questa tecnica si basa sulla raccolta, da parte di un rilevatore, di una matrice di dati 2D.
Tale matrice è costruita a partire da considerazioni fatte sull'attenuazione dei raggi X che, diretti da una sorgente verso il
rilevatore, passano attraverso l'oggetto di studio, vagliandolo.
Facendo ruotare la sorgente e il rilevatore, lungo una traiettoria nota, molte matrici di dati vengono in genere raccolte.
Queste matrici 2D vengono poi trasferite in un calcolatore che ha il compito di ricostruire l'entità scansionata in 3D.
Questo consente di esaminare l'oggetto ricostruito, osservandone alcune caratteristiche della sua struttura interna non note a
priori, senza doverlo a tal scopo necessariamente disassemblare.

Nonostante sfruttata principalmente in ambito medico, la teoria su cui si basa la TC, in realtà, non fu ideata per scopi medici.
Infatti tale tecnica trova applicazione in numerosi altri ambiti oltre alla medicina, tra cui vari settori industriali, la
paleontologia, l'archeologia e numerosi altri campi.

Il principio su cui si basa ha origine grazie al matematico austriaco Johann Radon, che nel 1917 dimostrò la possibilità di
ricostruire un oggetto tridimensionale mediante un numero infinito di proiezioni bidimensionali dell'oggetto stesso.
Dal punto di vista matematico, questo rappresenta un problema inverso, cioè si determinano le cause (immagine 3D) a partire dagli
effetti osservati (proiezioni 2D).

In questa analisi viene invece affrontato il problema diretto: partendo dal corpo studiato si devono perciò determinare le
proiezioni che una TC potrebbe generare.
Gli obiettivi di questo lavoro sono:
\begin{itemize}
  \item Implementare una versione CUDA di un certo algoritmo di proiezione proposto da Siddon \cite{Siddon1984} partendo da
        una sua implementazione parallela in C che sfrutta OpenMP \cite{Colletta2025}.
  \item Analizzare le prestazioni della versione CUDA per GPU e confrontarle con la versione parallela per CPU già nota.
\end{itemize}
\end{abstract}

\fancypagestyle{fancy}{\fancyfoot[C]{\thepage}}
\fancypagestyle{plain}{\fancyfoot[C]{\thepage}}
\pagestyle{fancy}
\pagenumbering{roman}
\tableofcontents
\clearpage

\fancypagestyle{fancy}{\fancyfoot[C]{\thepage\ di \pageref{LastPage}}}
\fancypagestyle{plain}{\fancyfoot[C]{\thepage\ di \pageref{LastPage}}}
\pagestyle{fancy}
\newpage
\pagenumbering{arabic}

% Chapter 1
\chapter{Introduzione}

\dots

\appendix

\chapter{Conclusioni}

\dots

\printbibliography
\thispagestyle{empty}

\end{document}
